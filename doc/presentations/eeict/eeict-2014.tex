\documentclass{beamer}

% Used packages

\usepackage[czech]{babel}
\usepackage[utf8]{inputenc}
\usepackage{multirow}
\usepackage{graphicx}
\usepackage{amsmath,bm,times}
\usepackage{tikz}
\usepackage{verbatim}
\usepackage{listings}
\usepackage{color}

\useoutertheme{infolines}
\usetheme{Copenhagen}
\mode<presentation>

% informace pouzite pro titulky

\title[Rozhodovací procedura pro logiku WS$k$S]{Rozhodovací procedura pro logiku
WS$k$S}
\subtitle{EEICT}
\author[T. Fiedor]{bc. Tomáš Fiedor}
\date{24. květen 2014}
\institute[vedoucí: Lengál]{pod vedením Ing. Ondřeje Lengála}

\begin{document}

\setbeamertemplate{footline}[infolines theme]

  \begin{frame}[plain]
    \titlepage
  \end{frame}

  \begin{frame}{Proč se vůbec zabývat logikami?}
  
  \end{frame}	

  \begin{frame}{WS$k$S}
  
  \end{frame}
  
  \begin{frame}{Rozhodování WS$k$S pomocí DFA}
  
  \end{frame}
  
  \begin{frame}{Srovnání přístupů přes DFA a NFA}
  
  \end{frame}
  
  \begin{frame}{Rozhodování WS$k$S pomocí NFA}
  
  \end{frame}
  
  \begin{frame}{Shrnutí}
  
  \end{frame}

\end{document}
