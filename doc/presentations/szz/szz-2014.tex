\documentclass{beamer}

% Used packages

\usepackage[czech]{babel}
\usepackage[utf8]{inputenc}
\usepackage{multirow}
\usepackage{graphicx}
\usepackage{amsmath,bm,times}
\usepackage{tikz}
\usepackage{verbatim}
\usepackage{listings}
\usepackage{color}

\useoutertheme{infolines}
\usetheme{Copenhagen}
\mode<presentation>

% informace pouzite pro titulky

\title[Rozhodovací procedura pro logiku WS$k$S]{Rozhodovací procedura pro logiku
WS$k$S}
\subtitle{Státní zavěřečná zkouška}
\author[T. Fiedor]{bc. Tomáš Fiedor}
\date{??. květen 2014}
\institute[vedoucí: Lengál]{pod vedením Ing. Ondřeje Lengála}

\begin{document}

\setbeamertemplate{footline}[infolines theme]

  \begin{frame}[plain]
    \titlepage
  \end{frame}
	
	\section{WS$k$S}
	\subsection{Úvod}
	\begin{frame}{WS$k$S logika}
     \begin{itemize}
       \item Logiky = vhodný prostředek pro specifikaci systému
       \item WS$k$S je logika
       \begin{itemize}
         \pause
         \item \emph{druhého řádu} $\Rightarrow$ možnost kvantifikace přes
         relace;
         \pause
         \item \emph{monadická} $\Rightarrow$ relace jsou unární, tzn. množiny a
         nikoliv např. funkce;
         \pause
         \item \emph{slabá} $\Rightarrow$ tyto množiny jsou konečné;
         \pause
         \item \emph{s $k$ následníky} $\Rightarrow$ možnost vyjádření stromových
         struktur.
         \pause
       \end{itemize}
       \item Příklad formule: $\forall x. x \in Z \Leftrightarrow (x \in X
       \wedge x \in Y)$
       \pause
       \item Proč WS$k$S?
       \pause
       \begin{itemize}
         \item Dobrá vyjadřovací síla;
         \pause
         \item Je rozhodnutelná. 
       \end{itemize}
         \pause
         \item Složitost problému splnitelnosti WS$k$S formule je ve třídě
         NONELEMENTARY.
         \pause
         \begin{itemize}
           \item[$\Rightarrow$] nutnost použití heuristik pro praktické řešení
           problému.
         \end{itemize}
     \end{itemize}
	\end{frame}
	
	\subsection{Rozhodovací procedura}
	\begin{frame}{Rozhodování WS$k$S pomocí stromových automatů}
	 \begin{itemize}
	   \item Rozhodovací procedura = je formule $\phi$ \textcolor{blue}{platná},
	   \textcolor{magenta}{neplatná, ale splnitelná}, nebo
	   \textcolor{red}{nesplnitelná}?
	 \end{itemize}
	 \begin{enumerate}
	   \pause
	   \item Převod $\phi$ do zjednodušené syntaxe
	   \begin{itemize}
	     \item[$\Rightarrow$] $\vee, \neg, \exists X, Sing, \subseteq, =$
	   \end{itemize}
	   \pause
	   \item Indukcí nad strukturou formule $\phi$:
	   \begin{enumerate}
	   \pause
	     \item Konstrukce elementárních automatů pro atomické podformule $\phi$
	   \pause
	     \item Indukční krok:
	     \begin{itemize}
	   \pause
	       \item $\phi_1 \vee \phi_2 \Rightarrow$ sjednocení automatů
	       $\mathcal{A}_{\phi_1}$ a $\mathcal{A}_{\phi_2}$
	   \pause
	       \item $\phi = \neg\phi_1 \Rightarrow$ komplementace automatu
	       $\mathcal{A}_{\phi_1}$
	   \pause
	       \item $\phi = \exists X_i.\phi_1 \Rightarrow$ $i$-tá projekce automatu
	       $\mathcal{A}_{\phi_1}$
	   \pause
	     \end{itemize}
	   \end{enumerate}
	   \item Testování jazyka $L_{\mathcal{A}_\phi}$
	 \end{enumerate}
	\end{frame}
	
	\begin{frame}{MONA}
	 \begin{itemize}
	  \item Implementace rozhodovací procedury pro WS1S a WS2S
	  \item Využívá deterministické automaty
	  \item Implementace řady heuristik
	  \begin{itemize}
	   \pause
	    \item[$\Rightarrow$] Reprezentace přechodové relace pomocí
	    \textcolor{blue}{binarních rozhodovacích diagramu} (BDD)
	   \pause
	    \item[$\Rightarrow$] Reprezentace formule pomocí
	    \textcolor{blue}{orientovaných acyklických grafů} (DAG)
	   \pause
	    \item[$\Rightarrow$] Redukce formule
	   \pause
	    \item[$\Rightarrow$] Cachování 
	  \end{itemize}
	   \pause
	   \item Nevýhoda nástroje MONA:
	   \begin{itemize}
	  \item Při zavedení nedeterminismu nastává okamžitá determinizace
	  \begin{itemize}
	    \pause
	    \item[$\Rightarrow$] ztráta informací o původních stavech
	  \end{itemize}
	  \end{itemize}
	 \end{itemize}
	\end{frame}
	
	\section{Náplň práce}
	\begin{frame}{Náplň práce}
	 \begin{itemize}
	   \item Návrh a implementace rozhodovací procedury založené na
	   nedeterministických automatech
	   \pause
	   \item Hledání (ne)přijímajících stavů za běhu
	   \begin{itemize}
	     \item Využití algoritmu založeného na anti-chainech
	   \pause
	   \end{itemize}
	   \item Efektivní manipulace se stromovými automaty = knihovna VATA
	   \pause
	   \item Tvorba hierarchické rodiny automatů
	 \end{itemize}
	\end{frame}
	
    \begin{frame}{Pokračování práce}
    \begin{enumerate}
      \item Navržení rozhodovací procedury pro WS$k$S založené na NTA
      \item Důkaz korektnosti této procedury
      \item Implementace navržené metody
      \item Srovnání výkonu s nástrojem MONA
    \end{enumerate}
  \end{frame}
\end{document}
