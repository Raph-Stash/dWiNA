\documentclass{eeict}
\inputencoding{utf8}
\usepackage[bf]{caption2}


\newtheorem{example}{Example}[section]
%--------------------------------------------------------------------------------

\title{Decision procedure for WS$k$S}
\author{Tomáš Fiedor}
\programme{Master Degree Programme 2, FIT BUT}
\emails{xfiedo01@stud.fit.vutbr.cz}

\supervisor{Ondřej Lengál}
\emailv{ilengal@fit.vutbr.cz}

\abstract{Various types of logics are often used as means for a formal
specification of systems. The weak monadic second-order logic with $k$
successors (WS$k$S) is one of these logics with quite high expressivity, yet
still is decidable. Although the complexity of checking satisfiability of
WS$k$S formula is not even in ELEMENTARY class, there are some approaches to
this problem that perform well in practice. With a recently developed
techniques for efficient manipulation of non-deterministic tree auotomata we
want to design and implement a decision procedure for WS$k$S based on
non-deterministic tree automata.}
\keywords{formal verification, tree automata, WS$k$S, decision procedures}

\begin{document}

\maketitle

%-------------------------------------------------------------------------------
\selectlanguage{english}
\section{Introduction}

\section{WS$k$S}

The abbreviation WS$k$S stands for \emph{weak second-order monadic logic of $k$
successors}. This means that it is a logic that allows quantification over set
variables (second-order), which can only represent \emph{finite} sets (weak) of
elements and \emph{not functions} (monadic) over a universe of discourse where
every element has $k$ successors and can therefore express a tree structure
(for $k \geq 2$).

A WS$k$S \emph{term} is either an empty constant $\epsilon$, a first-order
variable symbol written in lower-case letters (e.g. $x$, $y$, \ldots) or an
unary symbol from $\{1,\ldots,n\}$ written in postfix notation. 

Then \emph{atomic formulae}, for terms $s$ and $t$ are either the equality $s =
t$, inequalities $s \leq t$ and $s \geq t$ or the membership constaint $t \in
X$, for some second-order variable $X$.

The WS$k$S \emph{formula} is then built out of these atomic formulae using the
classical logical connectives $\wedge, \vee, \neg, \Leftarrow, \Leftrightarrow$
and quantifiers $\exists x, \forall x$ and $\exists X, \forall X$ for
quantification over first-order variables and second-order variables
respectively.

\begin{example} Simple WS$k$S formula denoting that for every element of set $X$
there exists its successor in $X$ $$ \forall x. \exists y. x \in X \wedge (x = y
+ 1 \Rightarrow y \in X) $$
\end{example}

We say that WS$k$S formula $\phi$ is in \emph{restricted syntax} if it does not
contain any first-order variables and it is build only over formulae $X
\subseteq Y$, $Sing(X)$, $X = Yi$ and $X = \epsilon$ using logical connectives
$\vee$, $\neg$ and $\exists$. For every WS$k$S formulae there exists a
translation to equivalent formula in restricted syntax \cite{tata}.

\section{Deciding WS$k$S}



\section{Conclussion}

%------------
% Citace
%
\begin{thebibliography}{9}
  \bibitem{tata}
\end{thebibliography}

\end{document}
