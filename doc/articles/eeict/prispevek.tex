\documentclass{eeict}
\inputencoding{utf8}
\usepackage[bf]{caption2}
\usepackage{amssymb}
\usepackage{amsmath}
\usepackage{dsfont}


\newtheorem{example}{Example}[section]
%--------------------------------------------------------------------------------

\title{Decision procedure for WS$k$S}
\author{Tomáš Fiedor}
\programme{Master Degree Programme 2, FIT BUT}
\emails{xfiedo01@stud.fit.vutbr.cz}

\supervisor{Ondřej Lengál}
\emailv{ilengal@fit.vutbr.cz}

\abstract{Various types of logics are often used as means for a formal
specification of systems. The weak monadic second-order logic with $k$
successors (WS$k$S) is one of these logics with quite high expressivity, yet
still is decidable. Although the complexity of checking satisfiability of
WS$k$S formula is not even in ELEMENTARY class, there are some approaches to
this problem that perform well in practice. With a recently developed
techniques for efficient manipulation of non-deterministic tree auotomata we
want to design and implement a decision procedure for WS$k$S based on
non-deterministic tree automata.}
\keywords{formal verification, tree automata, WS$k$S, decision procedures}

\begin{document}

\maketitle

%-------------------------------------------------------------------------------
\selectlanguage{english}
\section{Introduction}

\section{WS$k$S}

The abbreviation WS$k$S \cite{wsks} stands for \emph{weak second-order monadic
logic of $k$ successors}. This means that it is a logic that allows quantification over set
variables (second-order), which can only represent \emph{finite} sets (weak) of
elements and \emph{not functions} (monadic) over a universe of discourse where
every element has $k$ successors and can therefore express a tree structure
(for $k \geq 2$).

A WS$k$S \emph{term} is either an empty constant $\epsilon$, a first-order
variable symbol written in lower-case letters (e.g. $x$, $y$, \ldots) or an
unary symbol from $\{1,\ldots,n\}$ written in postfix notation. 

Then \emph{atomic formulae}, for terms $s$ and $t$ are either the equality $s =
t$, inequalities $s \leq t$ and $s \geq t$ or the membership constaint $t \in
X$, for some second-order variable $X$.

The WS$k$S \emph{formula} is then built out of these atomic formulae using the
classical logical connectives $\wedge, \vee, \neg, \Leftarrow, \Leftrightarrow$
and quantifiers $\exists x, \forall x$ and $\exists X, \forall X$ for
quantification over first-order variables and second-order variables
respectively.

\begin{example} Simple WS$k$S formula denoting that for every element of set $X$
there exists its successor in $X$ $$ \forall x. \exists y. x \in X \wedge (x = y
+ 1 \Rightarrow y \in X) $$
\end{example}

We say that WS$k$S formula $\phi$ is in \emph{restricted syntax} if it does not
contain any first-order variables and it is build only over formulae $X
\subseteq Y$, $Sing(X)$, $X = Yi$ and $X = \epsilon$ using logical connectives
$\vee$, $\neg$ and $\exists$. For every WS$k$S formulae there exists a
translation to equivalent formula in restricted syntax \cite{tata}.

\section{Deciding WS$k$S}

A decision procedure is an algorithm that given a formula $\phi$ returns VALID
if $\phi$ is valid, SAT if $\phi$ is invalid but satisfiable (additionally
yielding two assignemnts $\mathcal{M}_{true}$ and $\mathcal{M}_{false}$ such
that $\mathcal{M}_{true} \vDash \phi$ and $\mathcal{M}_{false}\not\nvDash \phi$)
and UNSAT if $\phi$ is unsatisfiable.

Decision procedures for WS$k$S make use of close link of formulae and automata,
i.e. given formula is transformed to correspondent automata and its language is
further examined.

\subsection{Deciding WS$k$S using deterministic automata}

One of the tools for deciding WS$k$S, MONA \cite{mona}, constructs automaton
$\mathcal{A}_\phi$ for given formula $\phi$	by structural induction of formula.
For each atomic subformulae we construct a correspondent automaton. Then as an
induction step, by considering only logical connectives from restricted syntax,
we construct for $\vee$, $\neg$ and $\exists$, union of automata, complement of
automaton and $i$-th projection of automaton respectively.

\begin{figure}
 \begin{center}
  
 \end{center}
 \caption{Automaton corresponding to example formula $\phi$}
\end{figure}

\subsection{Deciding WS$k$S using non-determinsitic automata}

Despite having good computational results in practice, due to using of
determinsitic automata every time a non-determinism is introduced then automaton
is determinised and thus information about the original states is forgotten.

So such an aproach has issues with extensive usage of automaton complementation
and since currently there is no known tree automaton complementation technique
better than bottom-up determinization of automaton with complementation of the
final states heavy optimizations and heuristics had to be used in MONA for
achieving good results.

We propose that it is not necessary to construct the automaton representing all
models of $\phi$. Instead the automaton and search for an accepting or
non-accepting state can be done \emph{on-the-fly} and exploit recent development
in algorithms using non-deterministic automata instead of deterministic ones. 

Technique used for deciding WS$k$S is similar to principle of antichains.

Given a WS$k$S formula $\phi$ we transform it to the formula in 
\emph{existantially-quantified prenex normal form} $\psi =
\exists\mathcal{X}_{m+1}\neg\exists\mathcal{X}_m\ldots\neg\exists\mathcal{X}_2\neg\exists\mathcal{X}_1.\pi(\mathds{X})$.
We then create a hierarchical family of WS$k$S formulae $\Phi =
\{\phi_0,\ldots,\phi_m\}$ where $\phi_0 = \pi$ and for all $0 \leq i \leq m-1$
it holds that $\phi_{i+1} = \neg\exists\mathcal{X}_{i+1}\phi_i$. 

\section{Conclusion}

We proposed a new decision procedure of WS$k$S logic that uses non-deterministic
automata instead of deterministic ones used in classical aproach, f.e. in tool
MONA \cite{mona}. This different approach makes use of a recent developments in
fields of non-deterministic automata algorithms like universality checking or
language inclusion, allowing us to search for a rejecting or accepting states
on-the-fly without constructing automaton corresponding to the given formula at
all, possible yielding better computational results for some types of formulae.

%------------
% Citace
%------------
\begin{thebibliography}{9}
  \bibitem{mona} MONA: Web pages of MONA. [online] Available on:
  http://www.brics.dk/mona/, Last Visited January 2014.
  \bibitem{tata}Comon, H. et al.: Tree automata techniques and applications.
  2007, release October, 12th 2007.
  \bibitem{wsks}Büchi, J. R.: Weak second-order arithmetic and finite automata.
  \emph{Mathematical Logic Quarterly}, 6(1-6):66-92, 1960.
\end{thebibliography}

\end{document}
